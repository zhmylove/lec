% Made by: KorG
% vim: ft=tex cc=79 ts=3 sw=3 et
\documentclass[hyperref={unicode=true}]{beamer}
\usepackage[T2A]{fontenc}       % fonts
\usepackage[utf8]{inputenc}     % UTF-8
\usepackage[english,russian]{babel}     % russian
\usepackage{cmap}               % russian search in pdf
\usepackage{droid}		        % Droid font
\usepackage{float}		        % essential for [H]
\usepackage{indentfirst}        % first string indention
\usepackage{graphicx}           % graphics
\usepackage{ltxtable}           % tables
\usepackage{amsmath}            % math
\usepackage{nccmath}            % math
\usepackage{amsfonts}           % math fonts
\usepackage{amssymb}            % math symb
\usepackage{color}
\usepackage{xcolor}
\definecolor{light-gray}{gray}{0.9}
\usepackage{multirow}
\usepackage{tabularx}
\usepackage{placeins}
\usepackage{totcount}
\usepackage{soul}               % \so{} & \ul{} - source and underline
\usepackage{soulutf8}           % UTF-8 for soul
\usepackage{verbatim}           % \verb{} and verbatim environment
\usepackage{listings}           % source code `bl
\usepackage{totcount}
\usepackage{pbox}
\usepackage{rotating}
\lstset{
   escapeinside={\#@}{@},
   extendedchars=\true,
   numbers=none,
   inputencoding=utf8,
   keepspaces=true,
   basicstyle=\large\ttfamily,
   backgroundcolor=\color{light-gray},
   tabsize=3,
   breaklines=true,
   postbreak=\raisebox{0ex}[0ex][0ex]{\ensuremath{
      \color{red}\hookrightarrow\space}
   }
}

\usebackgroundtemplate{
   \begin{picture}(0,260)
      \minipage{0.9\textwidth}
      \includegraphics[width=0.6\textwidth]{ifmo.jpg} %TODO check file path
      \endminipage
      \hfill \hspace*{60px}
      \ifnum\value{framenumber}>1 \Large \insertpagenumber \fi
   \end{picture}
}

\setbeamertemplate{navigation symbols}{}

\title{\LARGE Материалы к лекциям \\ по основам \\
   системного программирования \vspace{2em}}
\author{Жмылёв Сергей Александрович \vspace{-1em}}
\date{Осень 2018}

\begin{document} \Large

\begin{frame} \titlepage \end{frame}

\regtotcounter{section}
\setbeamertemplate{section in toc}{\inserttocsection}
\setbeamercolor{section in toc}{fg=black}
\begin{frame}[allowframebreaks]{Структура курса}
\fontsize{1em}{-1em}\selectfont
\vspace{1em} \tableofcontents[sections=1-8] \vspace{2em}
\ifnum\totvalue{section}>8 \framebreak
\vspace{1em} \tableofcontents[sections=9-16]  \vspace{2em} \fi
\ifnum\totvalue{section}>16 \framebreak
\vspace{1em} \tableofcontents[sections=17-24] \vspace{2em} \fi
\ifnum\totvalue{section}>24 \framebreak
\vspace{1em} \tableofcontents[sections=25-32] \vspace{2em} \fi
\ifnum\totvalue{section}>32 \framebreak
\vspace{1em} \tableofcontents[sections=33-40] \vspace{2em} \fi
\ifnum\totvalue{section}>40 \framebreak
\vspace{1em} \tableofcontents[sections=41-48] \vspace{2em} \fi
\ifnum\totvalue{section}>48 \framebreak
\vspace{1em} \tableofcontents[sections=49-56] \vspace{2em} \fi
\ifnum\totvalue{section}>56 \framebreak
\vspace{1em} \tableofcontents[sections=57-64] \vspace{2em} \fi
\ifnum\totvalue{section}>64 \framebreak
\vspace{1em} \tableofcontents[sections=65-72] \vspace{2em} \fi
\ifnum\totvalue{section}>72 \framebreak
\vspace{1em} \tableofcontents[sections=73-80] \vspace{2em} \fi
\ifnum\totvalue{section}>80 \framebreak
\vspace{1em} \tableofcontents[sections=81-88] \vspace{2em} \fi
\ifnum\totvalue{section}>88 \framebreak
\vspace{1em} \tableofcontents[sections=89-96] \vspace{2em} \fi
\ifnum\totvalue{section}>96 \framebreak
\vspace{1em} \tableofcontents[sections=97-104] \vspace{2em} \fi
\end{frame}

\newcommand{\iframe}[1]{
\section{#1}\begin{frame}[fragile]{#1}\par\vspace{-1em}
}
\newcommand{\pframe}[1]{
\begin{frame}[fragile]{#1}\par\vspace{-1em}
}

%%%%%%%%%%%%%%%%%%%%%%%%%%%%%%%%%%%%%%%%%%%%%%%%%%%%%%%%%%%%%%%%%%%%%%%%%%%%%%%

\iframe{Системное программное обеспечение:} 
\begin{itemize}
   \setlength\itemsep{2em}
      \item Языки СПО
      \item Системные вызовы
      \item Ввод-вывод
      \item Потоки и процессы
   \end{itemize}	
\end{frame}

\iframe{Программный интерфейс} 
   \begin{center}
      Любая программа принимает аргументы и
      переменные окружения
   \end{center}	
   \vspace{1cm} Код возврата – число, отображающее
   корректность завершения
\end{frame}

\iframe{Структура программ} 
\begin{lstlisting}
int main(
   int argc,
   char *argv[],
   #@\textcolor{gray}{char *envp[]} @
) {
   /* ... */

   return 0;
}
\end{lstlisting}
\end{frame}

\pframe{Структура программ} 
\begin{lstlisting}
#include <stdlib.h>

int main(
   int argc,
   char *argv[],
   #@\textcolor{gray}{char *envp[]} @
) {
   /* ... */

   return EXIT\_SUCCES;
}
\end{lstlisting}
\end{frame}

\iframe{Компиляция программ} 
\begin{lstlisting}
# gcc -c program.c
# gcc -o program program.o

# gcc -o program program.c

# cc -o program program.c
\end{lstlisting}
\end{frame}

\iframe{Код возврата} 
\begin{lstlisting}
# rm -f /etc/passwd 2<&-
# echo $?
1
# echo Hello, world!
Hello, world!
# echo $?
0
\end{lstlisting}
\end{frame}

\pframe{Популярная ошибка} 
   \begin{center}
      Использование void main() - недопустимо!
   \end{center}
   \lstinputlisting[language=C]{listings/e2.c}
\end{frame}

\iframe{Роль системы} 
   \begin{itemize}
   \setlength\itemsep{1em}
      \item Многозадачность;
      \item Виртуализация памяти;
      \item Управление устройствами;
      \item Обработка прерываний;
      \item Расширение набора операций, доступных программам.
   \end{itemize}
\end{frame}

\iframe{Системные вызовы}
   \begin{itemize}
   \setlength\itemsep{1em}
      \item Обращение к функции ядра системы
      \item Использование аппаратного обеспечения через единый API
      \item Имеют интерфейс libc
      \item Имеют привилегии ядра операционной системы
   \end{itemize}
\end{frame}	

\iframe{Обращение к системе} 
   \lstinputlisting[language=C]{listings/e3.c}
\end{frame}			

\iframe{Обращение к системе} 
   \lstinputlisting[language=C]{listings/e4.c}
\end{frame}	

\iframe{Файл} 
   \begin{center}
      Что такое файл?
   \end{center}
   \vspace{2cm} Everything is a file!

   \textit{Кроме потоков и ядра}
\end{frame}	

\iframe{Дескриптор файл} 
   \begin{itemize}
      \item Это целое неотрицательное число, абстрагирующее процессы от файлов, с которыми они работают.
   \end{itemize}
\end{frame}	

% vim: ft=tex cc=79 ts=3 sw=3 et
% 
\pframe{\Rus{Благодарности}\Eng{Special thanks to}}

\begin{itemize}
\item \Rus{Афанасьев Дмитрий Борисович}\Eng{Dmitry Borisovich Afanasyev}
\item \Rus{Горская Александра Андреевна}\Eng{Aleksandra Andreevna Gorskaia}
\item \Rus{Ховалкина Ксения Николаевна}\Eng{Ksenia Nikolaevna Khovalkina}
\item \Rus{Киреев Валерий Юрьевич}\Eng{Valery Yuryevich Kireev}
% \Eng{\item Nikita Andreevich Tomilov} % NT: am I worthy of this? ;)
\item \Rus{и многие другие...}\Eng{...and others}
\end{itemize}

\end{frame}
 % благодарности
%%%%%%%%%%%%%%%%%%%%%%%%%%%%%%%%%%%%%%%%%%%%%%%%%%%%%%%%%%%%%%%%%%%%%%%%%%%%%%%
\usebackgroundtemplate{}
\begin{frame}[fragile]
   \vspace{3em}
   \centering \LARGE Спасибо за внимание

   \vspace{4em}
   \begin{tabular}{c}
      \begin{lstlisting}[basicstyle=\tiny]
      # perl '-es!!),-#(-.?{<>-8#=..#<-*}>;*7-86)!;y!#()-?{}!\x20/`-v;<!;s++$_+ee'
      \end{lstlisting} 
   \end{tabular}
\end{frame}

\end{document}
