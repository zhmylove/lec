% vim: ft=tex cc=79 ts=3 sw=3 et
% 
% iframe = master frame; pframe = slave frame
\iframe{Перенаправление ввода-вывода}

\begin{tabular}{ll}
\lstinline{[1]> file } &  очистка + запись в начало \\
\lstinline{[1]>| file} &  очистка + запись в начало, \\
  & игнорируя noclobber \\
\lstinline{[1]>> file} &  запись в конец \\
\lstinline{[0]< file } &  чтение с начала \\
\lstinline{[0]<> file} &  чтение + запись в начало \\
   \lstinline{[0]<&n2 } &  скопировать n2 -> n1 \\
\lstinline{[0]<&-   } &  закрыть дескриптор \\
\lstinline{[1]>&n2 } &  скопировать n2 -> n1 \\
\lstinline{[1]>&-   } &  закрыть дескриптор
\end{tabular}

\end{frame}

\pframe{Перенаправление <>}

Открывает stdin на чтение и запись, позиция внутри файла устанавливается в
начало (0).

\begin{lstlisting}
$ echo 123 > file
$ perl -le '
sysread \*STDIN, my $x, 1; 
print $x; 
syswrite \*STDIN, "x", 1;
' <> file
===> 1
$ cat file
===> 1x3
\end{lstlisting}

\end{frame}

\pframe{Перенаправление ввода-вывода}

\begin{lstlisting}
[n]<<[-] разделитель 
   ... 
   буквы $name
   ... 
   разделитель 
   ... 
разделитель

\end{lstlisting}

\begin{lstlisting}
$ pfiles 13666
...
3: S_IFREG mode:0600 dev:310,2 ino:4195683897 uid:378 gid:10 size:0
   O_RDWR|O_CREAT|O_EXCL|O_LARGEFILE
   /tmp/sh270320
\end{lstlisting}

\end{frame}
