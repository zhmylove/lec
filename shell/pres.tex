% Made by: KorG
% vim: ft=tex cc=79 ts=3 sw=3 et
\documentclass{beamer}
\usepackage[T2A]{fontenc}       % fonts
\usepackage[utf8]{inputenc}     % UTF-8
\usepackage[english,russian]{babel}     % russian
\usepackage{cmap}               % russian search in pdf
\usepackage{pscyr}		        % PSCyr
\usepackage{float}		        % essential for [H]
\usepackage{indentfirst}        % first string indention
\usepackage{graphicx}           % graphics
\usepackage{ltxtable}           % tables
\usepackage{amsmath}            % math
\usepackage{nccmath}            % math
\usepackage{amsfonts}           % math fonts
\usepackage{amssymb}            % math symb
\usepackage{color}
\usepackage{xcolor}
\definecolor{light-gray}{gray}{0.9}
\usepackage{multirow}
\usepackage{tabularx}
\usepackage{placeins}
\usepackage{totcount}
\usepackage{soul}               % \so{} & \ul{} - source and underline
\usepackage{soulutf8}           % UTF-8 for soul
\usepackage{verbatim}           % \verb{} and verbatim environment
\usepackage{listings}           % source code `bl
\lstset{
   escapeinside={\#@}{@},
        extendedchars=\true,
        numbers=none,
        inputencoding=utf8,
        keepspaces=true,
        basicstyle=\large\ttfamily,
        backgroundcolor=\color{light-gray},
        tabsize=3,
        breaklines=true,
        postbreak=\raisebox{0ex}[0ex][0ex]{\ensuremath{
           \color{red}\hookrightarrow\space}
        }
     }

\usebackgroundtemplate{
   \begin{picture}(0,260)
      \minipage{0.9\textwidth}
      \includegraphics[width=0.6\textwidth]{ifmo.jpg} %TODO check file path
      \endminipage
      \hfill \hspace*{60px}
      \ifnum\value{framenumber}>1 \Large \insertpagenumber \fi
   \end{picture}
}

\setbeamertemplate{navigation symbols}{}

\title{\LARGE Слайды к лекционному курсу \\ по командным интерпретаторам \\
   и разработке скриптов: \\ sh, ksh, bash, perl}
\author{Жмылёв С.~А.}
\date{Осень 2018}

\begin{document} \Large

\begin{frame} \titlepage \end{frame}

\newcounter{TocNum}
\newcounter{TocEnum}
\begin{frame}[fragile]{Структура курса (\stepcounter{TocNum}\arabic{TocNum})}
\begin{enumerate}

\item Взаимодействие с системой
\item Философия UNIX: текстовые потоки
\item Потоки ввода-вывода
\item Командный интерпретатор
\item Разновидности интерпретаторов
\item Командный файл (скрипт)
\item Комментарии
\item Команды: утилиты, функции, ...

\setcounter{TocEnum}{\value{enumi}}
\end{enumerate}
\end{frame}

%% BEGIN_FRAME
\begin{frame}[fragile]{Структура курса (\stepcounter{TocNum}\arabic{TocNum})}
\begin{enumerate} \setcounter{enumi}{\value{TocEnum}}

\item Перечень часто используемых утилит
\item Переменные
\item Условные операторы
\item Операторы цикла
\item (t)csh: история, где встречается, концептуальные и синтаксические отличия

\end{enumerate}
\end{frame}
%% END_FRAME

\begin{frame}[fragile]{Title} \begin{lstlisting}
# some code...
\end{lstlisting} 
\end{frame}

%%%%%%%%%%%%%%%%%%%%%%%%%%%%%%%%%%%%%%%%%%%%%%%%%%%%%%%%%%%%%%%%%%%%%%%%%%%%%%%
\usebackgroundtemplate{}
\begin{frame}
   \centering \LARGE Спасибо за внимание
\end{frame}

\end{document}
